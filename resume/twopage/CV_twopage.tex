%%%%%%%%%%%%%%%%%%%%%%%%%%%%%%%%%%%%%%%%%
% Medium Length Professional CV
% LaTeX Template
% Version 2.0 (8/5/13)
%
% This template has been downloaded from:
% http://www.LaTeXTemplates.com
%
% Original author:
% Rishi Shah 
%
% Important note:
% This template requires the resume.cls file to be in the same directory as the
% .tex file. The resume.cls file provides the resume style used for structuring the
% document.
%
%%%%%%%%%%%%%%%%%%%%%%%%%%%%%%%%%%%%%%%%%

%----------------------------------------------------------------------------------------
%	PACKAGES AND OTHER DOCUMENT CONFIGURATIONS
%----------------------------------------------------------------------------------------

\documentclass{resume} % Use the custom resume.cls style
\usepackage{titlesec}
\usepackage{titling}
\usepackage{array}
\usepackage{graphicx}
\usepackage{enumitem}
\usepackage{xcolor}
\usepackage{adjustbox}
\usepackage[colorlinks=true, urlcolor=blue, linkcolor=red, citecolor=green, anchorcolor=black]{hyperref}

\usepackage{amsmath,mathrsfs,amssymb}
\usepackage[left=0.6in,top=0.4in,right=0.6in,bottom=0.4in]{geometry} % Document margins
\newcommand{\tab}[1]{\hspace{.2667\textwidth}\rlap{#1}}
\newcommand{\itab}[1]{\hspace{0em}\rlap{#1}}
% 定义自定义颜色
\definecolor{myurlcolor}{RGB}{0,0,100}

% 应用自定义颜色
\hypersetup{
    urlcolor=myurlcolor % 使用自定义颜色
}

% \name{Yu-Jie Zhang} % Your name
% \address{Department of Complexity Science and Engineering, The University of Tokyo}
% \address{Email:\ \texttt{yujie.zhang@ms.k.u-tokyo.ac.jp}}
% \address{Website:\  \texttt{https://yujie-zhang96.github.io/}}


% 设置无页眉页脚
\pagestyle{empty}
% 设置标题格式
\titleformat{\section}{\bfseries\Large}{\thesection}{1em}{}

% 自定义命令以创建姓名和职位
\newcommand{\makecontact}[2]{
\noindent\begin{minipage}[t]{0.5\textwidth}
{\MakeUppercase\huge\bf #1}\\[0.5em]
{\large The University of Tokyo}
\end{minipage}%
\hfill
\begin{minipage}[t]{0.45\textwidth}
\raggedleft
\adjustbox{valign=c}{\includegraphics[height=1em]{email.png}} \href{mailto:#2}{#2} \\
\adjustbox{valign=c}{\includegraphics[height=1em]{web.png}} \href{https://yujie-zhang96.github.io/}{{yujie-zhang96.github.io}} \\
\end{minipage}
}


 % Your phone number and email

\begin{document}
\makecontact{Yu-Jie Zhang}{yujie.zhang@ms.k.u-tokyo.ac.jp}
% \begin{rSection}{Contact Information}
% % NOTE: Mind where the & separators and \\ breaks are in the following
% %       table.
% %
% % ALSO: \rcollength is the width of the right column of the table
% %       (adjust it to your liking; default is 1.85in).
% %
% % \newlength{\rcollength}\setlength{\rcollength}{1.3in}%
% %
% {Department of Computer Science and Technology} \hfill {email: zhangyj@lamda.nju.edu.cn}\\
% Nanjing University, Xianlin Campus \hfill {or zhangyj.gm@gmail.com}\\
% 163 Xianlin Avenue, Qixia District, Nanjing, China  \hfill {homepage: \href{http://www.lamda.nju.edu.cn/zhaop}{www.lamda.nju.edu.cn/zhangyj}}

% \begin{tabular}[t]{@{}p{\textwidth-\rcollength}p{\rcollength}}
% Nanjing University, Xianlin Campus Mailbox 603 & \textit{Phone:} +86-25-89680949 \\
% 163 Xianlin Avenue, Qixia District  & \textit{E-mail:} \href{mailto:zlj@nju.edu.cn}{zlj@nju.edu.cn}\\
% Nanjing 210023, China & \textit{WWW:} \href{http://cs.nju.edu.cn/zlj}{http://cs.nju.edu.cn/zlj}
% \end{tabular}
% \end{rSection}
%----------------------------------------------------------------------------------------
%	EDUCATION SECTION
%----------------------------------------------------------------------------------------

\begin{rSection}{Education}
{\bf The University of Tokyo}, Japan \hfill {October 2021 - Present}
\\ Ph.D. candidate, Complexity Science and Engineering \hfill {Supervisor: Prof. \href{http://www.ms.k.u-tokyo.ac.jp/sugi/index.html}{Masashi Sugiyama}}\vspace{2mm}
\\{\bf Nanjing University}, China \hfill { June 16, 2021} 
\\ M.Sc., Computer Science and Technology \hfill {Supervisor: Prof. \href{http://cs.nju.edu.cn/zhouzh}{Zhi-Hua Zhou}}\vspace{2mm}
\\{\bf Tongji University}, China \hfill{July 01, 2018}
\\ B.Sc., Electronic Science and Technology \hfill {GPA: 4.91/5.00, ranking 1/32} 

%Minor in Linguistics \smallskip \\
%Member of Eta Kappa Nu \\
%Member of Upsilon Pi Epsilon \\
\end{rSection}

\begin{rSection}{Research Interest}
My research focuses on developing machine learning techniques to learn with non-stationary and imperfect data, particularly from the following perspectives:	
\begin{itemize}
	\item Online learning in non-stationary environments 
	\item Classification with imperfect data 
	\item Contextual bandit with nonlinear reward
\end{itemize}
\end{rSection}

\begin{rSection}{Publications}
\noindent \underline{\textbf{Preprints}}
\begin{enumerate}[leftmargin=*]
	\item S. Chen, \textbf{Y.-J. Zhang}, W.-W. Tu, P. Zhao, and L. Zhang. Optimistic Online Mirror Descent for Bridging Stochastic and Adversarial Online Convex Optimization. In submission to JMLR, minor revision. 
	\item P. Zhao, \textbf{Y.-J. Zhang}, L. Zhang, and Z.-H. Zhou. Adaptivity and Non-stationarity: Problem-dependent Dynamic Regret for Online Convex Optimization. In submission to JMLR, minor revision.
	\item W. Wang, T. Ishida, \textbf{Y.-J. Zhang}, G. Niu, and M. Sugiyama. Learning with Complementary Labels Revisited: A Consistent Approach via Negative-Unlabeled Learning.	
\end{enumerate}
\noindent \underline{\textbf{Conference Publications}}
\begin{enumerate}[leftmargin=*]
	\item \textbf{Y.-J. Zhang} and M. Sugiyama. Online (Multinomial) Logistic Bandit: Improved Regret and Constant Computation Cost. In {Advances in Neural Information Processing Systems 36} (\textbf{NeurIPS}), 2023. [\textbf{Spotlight}]
	\item \textbf{Y.-J. Zhang}, Z.-Y. Zhang, P. Zhao, and M. Sugiyama. Adapting to Continuous Covariate Shift via Online Density Ratio Estimation. In {Advances in Neural Information Processing Systems 36} (\textbf{NeurIPS}), 2023.
	\item X.-Q. Cai, \textbf{Y.-J. Zhang}, C.-K. Chiang and M. Sugiyama. Imitation Learning from Vague Feedback. In {Advances in Neural Information Processing Systems 36} (\textbf{NeurIPS}), 2023.
	\item Y. Bai*, \textbf{Y.-J. Zhang*}, P. Zhao, M. Sugiyama, and Z.-H. Zhou. Adapting to Online Label Shift with Provable Guarantees. In {Advances in Neural Information Processing Systems 35} (\textbf{NeurIPS}), 2022.  (* equal contribution)
	\item Z.-Y. Zhang, Y.-Y. Qian, \textbf{Y.-J. Zhang}, Y. Jiang, Z.-H. Zhou. Adaptive Learning for Weakly Labeled Streams. In {Proceedings of the 28th ACM SIGKDD Conference on Knowledge Discovery and Data Mining} (\textbf{KDD}), 2022.
	\item \textbf{Y.-J. Zhang}, Y.-H. Yan, P. Zhao and Z.-H. Zhou. Towards Enabling Learnware to Handle Unseen Jobs. In {Proceedings of the 35th AAAI Conference on Artificial Intelligence} (\textbf{AAAI}), 2021.
	\item P. Zhao, \textbf{Y.-J. Zhang} and Z.-H. Zhou. Exploratory Machine Learning with Unknown Unknowns. In {Proceedings of the 35th AAAI Conference on Artificial Intelligence} (\textbf{AAAI}), 2021.
	\item \textbf{Y.-J. Zhang}, P. Zhao, L. Ma and Z.-H. Zhou. An Unbiased Risk Estimator for Learning with Augmented Classes. In {Advances in Neural Information Processing Systems 33} (\textbf{NeurIPS}), 2020.
	\item P. Zhao, \textbf{Y.-J. Zhang}, L. Zhang and Z.-H. Zhou. Dynamic Regret of Convex and Smooth Functions. In {Advances in Neural Information Processing Systems 33} (\textbf{NeurIPS}), 2020.
	\item \textbf{Y.-J. Zhang}, P. Zhao, and Z.-H. Zhou. A Simple Online Algorithm for Competing with Dynamic Comparators. In {Proceedings of the 36th Conference on Uncertainty in Artificial Intelligence} (\textbf{UAI}), 2020.
\end{enumerate}
\noindent \underline{\textbf{Journal Publications}}
\begin{enumerate}[leftmargin=*]
	\item P. Zhao, J.-W. Shan, \textbf{Y.-J. Zhang} and Z.-H. Zhou. Exploratory Machine Learning with Unknown Unknowns. Artificial Intelligence (\textbf{AIJ}), to appear, 2024.
\end{enumerate}
\end{rSection}

%--------------------------------------------------------------------------------
%    Projects And Seminars
%-----------------------------------------------------------------------------------------------
%----------------------------------------------------------------------------------------
%	TECHNICAL STRENGTHS SECTION
%----------------------------------------------------------------------------------------

% \begin{rSection}{Technical Strengths}

% \begin{tabular}{ @{} >{\bfseries}l @{\hspace{6ex}} l }
% Modeling and Analysis \ & AutoCad, Revit, StaadPro \\
% Software \& Tools & MS Office, Latex \\
% \end{tabular}

% \end{rSection}

%----------------------------------------------------------------------------------------
%	WORK EXPERIENCE SECTION
%----------------------------------------------------------------------------------------

%	EXAMPLE SECTION
%----------------------------------------------------------------------------------------

% \begin{rSection}{Academic Achievements} 
%  Runners up in B.G.Shirke Vidyarthi Competition for Innovative Project organized by Pune Construction Engineering Research Foundation in January 2018
% \item Won First Prize in Model Making Competition Organized by Symbiosis Institute of Technology, Pune.
% \end{rSection}

\begin{rSection}{Awards \& Honors}
\begin{itemize}[leftmargin=*]
\item Top Reviewer for NeurIPS 2023, 2023
\item Top Reviewer for UAI 2023, 2023
\item Top Reviewer for NeurIPS 2022, 2022
\item The University of Tokyo Fellowship, Tokyo, 2021
\item Outstanding Master Dissertation Award by Jiangsu Computer Society, Nanjing, 2021
\item Excellent Graduate of Nanjing University, Nanjing, 2021
\item National Graduate Scholarship for Master Student, MOE of PRC, 2020
\end{itemize}
\end{rSection}

\begin{rSection}{Academic Service}
\begin{itemize}[leftmargin=*]
\item Reviewer for Conference: NeurIPS (2021-2023), ICML (2022-2023), ICLR (2022-2024), AISTATS (2021-2024), UAI (2022-2023), AAAI (2021, 2024), IJCAI (2020-2023), ECAI (2020).
\item Reviewer for Journal: Journal of Machine Learning Research (JMLR), IEEE Transactions on Pattern Analysis and Machine Intelligence (TPAMI), Frontiers of Computer Science.
\end{itemize}

\end{rSection}


\end{document}
