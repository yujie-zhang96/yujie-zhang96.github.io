%%%%%%%%%%%%%%%%%%%%%%%%%%%%%%%%%%%%%%%%%
% Medium Length Professional CV
% LaTeX Template
% Version 2.0 (8/5/13)
%
% This template has been downloaded from:
% http://www.LaTeXTemplates.com
%
% Original author:
% Rishi Shah 
%
% Important note:
% This template requires the resume.cls file to be in the same directory as the
% .tex file. The resume.cls file provides the resume style used for structuring the
% document.
%
%%%%%%%%%%%%%%%%%%%%%%%%%%%%%%%%%%%%%%%%%

%----------------------------------------------------------------------------------------
%	PACKAGES AND OTHER DOCUMENT CONFIGURATIONS
%----------------------------------------------------------------------------------------

\documentclass{resume} % Use the custom resume.cls style
\usepackage{titlesec}
\usepackage{titling}
\usepackage{array}
\usepackage{graphicx}
\usepackage{enumitem}
\usepackage{xcolor}
\usepackage{adjustbox}
\usepackage[most]{tcolorbox}
\usepackage[colorlinks=true, urlcolor=blue, linkcolor=red, citecolor=green, anchorcolor=black]{hyperref}
\usepackage{pifont}
\usepackage{amsmath,mathrsfs,amssymb}
\usepackage[left=1in,top=0.6in,right=1in,bottom=0.6in]{geometry} % Document margins
\newcommand{\tab}[1]{\hspace{.2667\textwidth}\rlap{#1}}
\newcommand{\itab}[1]{\hspace{0em}\rlap{#1}}
% 定义自定义颜色
\definecolor{myurlcolor}{RGB}{0,0,130}
\definecolor{darkred}{RGB}{192,0,0}
\definecolor{myblue}{RGB}{11,139,238}
\linespread{1.2} % Increase line spacing by 1.5 times

% 应用自定义颜色
\hypersetup{
    urlcolor=myurlcolor % 使用自定义颜色
}

% \name{Yu-Jie Zhang} % Your name
% \address{Department of Complexity Science and Engineering, The University of Tokyo}
% \address{Email:\ \texttt{yujie.zhang@ms.k.u-tokyo.ac.jp}}
% \address{Website:\  \texttt{https://yujie-zhang96.github.io/}}


% 设置无页眉页脚
\pagestyle{empty}
% 设置标题格式
\titleformat{\section}{\bfseries\Large}{\thesection}{1em}{}

% 自定义命令以创建姓名和职位
\newcommand{\makecontact}[2]{
\noindent\begin{minipage}[t]{0.5\textwidth}
{\MakeUppercase\huge\bf #1}\\[0.5em]
{\large RIKEN AIP}
\end{minipage}%
\hfill
\begin{minipage}[t]{0.45\textwidth}
\raggedleft
\adjustbox{valign=c}{\includegraphics[height=1em]{email.png}} \href{mailto:#2}{#2} \\
\adjustbox{valign=c}{\includegraphics[height=1em]{web.png}} \href{https://yujie-zhang96.github.io/}{{yujie-zhang96.github.io}} \\
\end{minipage}
}



\begin{document}
\makecontact{\Large Yu-Jie Zhang}{yu-jie.zhang@riken.jp}
%----------------------------------------------------------------------------------------
%	EDUCATION SECTION
%----------------------------------------------------------------------------------------

\begin{rSection}{Education}
{\bf The University of Tokyo}, Japan \hfill {October 2021 - September 2024}
\\ Ph.D., Complexity Science and Engineering \hfill {Supervisor: Prof. \href{http://www.ms.k.u-tokyo.ac.jp/sugi/index.html}{Masashi Sugiyama}}\vspace{2.5mm}
\\{\bf Nanjing University}, China \hfill { June 16, 2021} 
\\ M.Sc., Computer Science and Technology \hfill {Supervisor: Prof. \href{http://cs.nju.edu.cn/zhouzh}{Zhi-Hua Zhou}}\vspace{2.5mm}
\\{\bf Tongji University}, China \hfill{July 01, 2018}
\\ B.Sc., Electronic Science and Technology \hfill {GPA: 4.91/5.00, ranking 1/32} 

\end{rSection}

\begin{rSection}{Work Experience \& Activities}
{\bf RIKEN Center for Advanced Intelligence Project}, Japan \hfill {January 2025- Now}
\\Postdoctoral Researcher \hfill {Supervisor: Prof. \href{http://www.ms.k.u-tokyo.ac.jp/sugi/index.html}{Masashi Sugiyama}}\vspace{2.5mm}
\\
{\bf The Institute for AI and Beyond}, Japan \hfill {April 2022- September 2024}
\\Research Assistant
\end{rSection}

\begin{rSection}{Research Interest}
I am generally interested in exploring the theoretical foundations of machine learning and developing methods with sound theoretical guarantees. Currently, I focus on developing provably adaptive and reliable methods for \emph{non-stationary and open-world environments}. My research topics include:
\begin{itemize}
	\item \textbf{Sequential Decision-making in Non-stationary Environments}: We develop online learning methods that adapt promptly to non-stationary environment with regret guarantees\vspace{-1mm}
	\begin{itemize}[label = {-}, left=- 2mm]
		\item \underline{\emph{Key Words:}} \emph{online optimization}, \emph{bandits}, \emph{reinforcement learning}, \emph{dynamic regret bound}.
	\end{itemize}\vspace{1.5mm}
	\item \textbf{Supervised Learning in Open-world Environments}: We develop reliable methods to learn with weak supervision and handle unknown classes with excess risk guarantees.\vspace{-1mm}
	\begin{itemize}[label = {-}, left = -2mm]
		\item \underline{\emph{Key Words:}} \emph{distribution shift}, \emph{weakly supervised learning}, \emph{unknown classes}, \emph{excess risk bound}. 
	\end{itemize}
\end{itemize}
\end{rSection}





\begin{rSection}{Publications}
\noindent {\textbf{Preprints}}
\begin{enumerate}[leftmargin=0.2in]
	\item Soichiro Nishimori, \underline{Yu-Jie Zhang}, Thanawat Lodkaew, Masashi Sugiyama. On Symmetric Losses for Robust Policy Optimization with Noisy Preferences. \emph{In submission}.
\end{enumerate}

\noindent {\textbf{Conference Publications}}
\begin{enumerate}[leftmargin=0.2in]

	\item	\underline{Yu-Jie Zhang}, Sheng-An Xu, Peng Zhao, Masashi Sugiyama. Generalized Linear Bandits: Almost Optimal Regret with One-Pass Update. \textbf{NeurIPS 2025}. 

	\item \underline{Yu-Jie Zhang}, Peng Zhao, and Masashi Sugiyama. Non-stationary Online Learning for Curved Losses: Improved Dynamic Regret via Mixability. \textbf{ICML 2025}. 
	
	\item Jing Wang, \underline{Yu-Jie Zhang}, Peng Zhao, and Zhi-Hua Zhou. Heavy-Tailed Linear Bandits: Huber Regression with One-Pass Update. \textbf{ICML 2025}.

	\item Yuting Tang, Yivan Zhang, Johannes Ackermann, \underline{Yu-Jie Zhang}, Soichiro Nishimori, Masashi Sugiyama. Recursive Reward Aggregation. \textbf{RLC 2025}.
	
	\item Long-Fei Li, \underline{Yu-Jie Zhang}, Peng Zhao, and Zhi-Hua Zhou. Provably Efficient Reinforcement Learning with Multinomial Logit Function Approximation. \textbf{NeurIPS 2024}.
	
	\item Yu-Yang Qian, Peng Zhao, \underline{Yu-Jie Zhang}, Masashi Sugiyama, Zhi-Hua Zhou. Efficient Non-stationary Online Learning by Wavelets with Applications to Online Distribution Shift Adaptation. \textbf{ICML 2024}
	
	\item Wei Wang, Takashi Ishida, \underline{Yu-Jie Zhang}, Gang Niu, and Masashi Sugiyama. Learning with Complementary Labels Revisited: A Consistent Approach via Negative-Unlabeled Learning. \textbf{ICML 2024}.
	 	 
	\item \underline{Yu-Jie Zhang} and Masashi Sugiyama. Online (Multinomial) Logistic Bandit: Improved Regret and Constant Computation Cost. \textbf{NeurIPS 2023} {\color{darkred}[\textbf{Spotlight}]}
	\item \underline{Yu-Jie Zhang}, Zhen-Yu Zhang, Peng Zhao, and Masashi Sugiyama. Adapting to Continuous Covariate Shift via Online Density Ratio Estimation. \textbf{NeurIPS 2023}.
	
	\item Xin-Qiang Cai, \underline{Yu-Jie Zhang}, Chao-Kai Chiang and Masashi Sugiyama. Imitation Learning from Vague Feedback. In {Advances in Neural Information Processing Systems 36} \textbf{NeurIPS 2023}.
	
	\item Yong Bai*, \underline{Yu-Jie Zhang*}, Peng Zhao, Masashi Sugiyama, and Zhi-Hua Zhou. Adapting to Online Label Shift with Provable Guarantees. \textbf{NeurIPS 2023}.  (* equal contribution)
	
	\item Zhen-Yu Zhang, Yu-Yang Qian, \underline{Yu-Jie Zhang}, Yuan Jiang, Zhi-Hua Zhou. Adaptive Learning for Weakly Labeled Streams. \textbf{KDD 2022}.
	
	\item \underline{Yu-Jie Zhang}, Yu-Hu Yan, Peng Zhao and Zhi-Hua Zhou. Towards Enabling Learnware to Handle Unseen Jobs. In {Proceedings of the 35th AAAI Conference on Artificial Intelligence} (\textbf{AAAI}), 2021.
	\item Peng Zhao, \underline{Yu-Jie Zhang} and Zhi-Hua Zhou. Exploratory Machine Learning with Unknown Unknowns. \textbf{AAAI 2021}.
	
	\item \underline{Yu-Jie Zhang}, Peng Zhao, Lanjihong Ma and Zhi-Hua Zhou. An Unbiased Risk Estimator for Learning with Augmented Classes. \textbf{NeurIPS 2020}.
	
	\item Peng Zhao, \underline{Yu-Jie Zhang}, Lijun Zhang and Zhi-Hua Zhou. Dynamic Regret of Convex and Smooth Functions. \textbf{NeurIPS 2020}.
	
	\item \underline{Yu-Jie Zhang}, Peng Zhao, and Zhi-Hua Zhou. A Simple Online Algorithm for Competing with Dynamic Comparators. \textbf{UAI 2020}.
\end{enumerate}
\vspace{2mm}
\noindent {\textbf{Journal Publications}}
\begin{enumerate}[leftmargin=0.2in]
	\setcounter{enumi}{16}
	\item Sijia Chen, \underline{Yu-Jie Zhang}, Wei-Wei Tu, Peng Zhao, and Lijun Zhang. Optimistic Online Mirror Descent for Bridging Stochastic and Adversarial Online Convex Optimization. Journal of Machine Learning Research (\textbf{JMLR}), 25(178):1−62, 2024.
	\item Peng Zhao, \underline{Yu-Jie Zhang}, Lijun Zhang, and Zhi-Hua Zhou. Adaptivity and Non-stationarity: Problem-dependent Dynamic Regret for Online Convex Optimization. Journal of Machine Learning Research (\textbf{JMLR}), 25(98):1−52, 2024.
	\item Peng Zhao, Jia-Wei Shan, \underline{Yu-Jie Zhang} and Zhi-Hua Zhou. Exploratory Machine Learning with Unknown Unknowns. Artificial Intelligence (\textbf{AIJ}), 327:104059, 2024.
\end{enumerate}
\end{rSection}

%--------------------------------------------------------------------------------
%    Projects And Seminars
%-----------------------------------------------------------------------------------------------
%----------------------------------------------------------------------------------------
%	TECHNICAL STRENGTHS SECTION
%----------------------------------------------------------------------------------------

% \begin{rSection}{Technical Strengths}

% \begin{tabular}{ @{} >{\bfseries}l @{\hspace{6ex}} l }
% Modeling and Analysis \ & AutoCad, Revit, StaadPro \\
% Software \& Tools & MS Office, Latex \\
% \end{tabular}

% \end{rSection}

%----------------------------------------------------------------------------------------
%	WORK EXPERIENCE SECTION
%----------------------------------------------------------------------------------------

%	EXAMPLE SECTION
%----------------------------------------------------------------------------------------

% \begin{rSection}{Academic Achievements} 
%  Runners up in B.G.Shirke Vidyarthi Competition for Innovative Project organized by Pune Construction Engineering Research Foundation in January 2018
% \item Won First Prize in Model Making Competition Organized by Symbiosis Institute of Technology, Pune.
% \end{rSection}

\begin{rSection}{Awards \& Honors}
\begin{itemize}[leftmargin=*]
\item Dean's Award for Outstanding Achievement (Doctoral Course), GSFS, UTokyo, 2025.	
\item AISTATS 2025 Best Reviewer, 2025
\item Top Reviewer for NeurIPS, 2023
\item Top Reviewer for UAI, 2023
\item Top Reviewer for NeurIPS, 2022
\item The University of Tokyo Fellowship, Tokyo, 2021
\item Outstanding Master Dissertation Award by Jiangsu Computer Society, Nanjing, 2021
\item Excellent Graduate of Nanjing University, Nanjing, 2021
\item National Graduate Scholarship for Master Student, MOE of PRC, 2020
\end{itemize}
\end{rSection}

\begin{rSection}{Academic Service}
\begin{itemize}[leftmargin=*]
\item \textbf{Reviewer for Conference}: NeurIPS (2021-2025), ICML (2022-2025), ICLR (2022-2025), AISTATS (2021-2025), UAI (2022-2024), AAAI (2021, 2024), IJCAI (2020-2023), ECAI (2020).
\item \textbf{Reviewer for Journal}: Journal of Machine Learning Research (JMLR), IEEE Transactions on Pattern Analysis and Machine Intelligence (TPAMI), Frontiers of Computer Science.
\end{itemize}

\end{rSection}


\end{document}
